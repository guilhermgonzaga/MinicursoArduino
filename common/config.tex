% SPDX-License-Identifier: CC0-1.0

\documentclass[smaller]{beamer} % aspectratio=,draft,handout,unknownkeysallowed


% PALETA DE CORES ======================


\definecolor{ArduinoBrown}{HTML}{8C7965}
\definecolor{ArduinoOrange}{HTML}{E47128}
\definecolor{ArduinoTeal}{HTML}{068183} % Alternativa: 00878F
\definecolor{ArduinoYellow}{HTML}{E5AD24}
\definecolor{ArduinoLightBrown}{HTML}{BCAC9A}
\definecolor{ArduinoLightGrey}{HTML}{ECF1F1}
\definecolor{ArduinoLightOrange}{HTML}{F29C2D}
\definecolor{ArduinoLightTeal}{HTML}{62AEB2} % Alternativa: 80CCCC


% TEMA =================================


\usetheme{focus}

\definecolor{main}{named}{ArduinoTeal}
\definecolor{background}{named}{ArduinoLightGrey}
\definecolor{alert}{named}{ArduinoLightOrange}
\definecolor{example}{named}{ArduinoBrown}

\setbeamercolor{block title}{bg=main!85!background,fg=background!50!white}
\setbeamercolor{block body}{bg=main!8!background,fg=main!20!black}
\setbeamercolor{block title alerted}{bg=alert,fg=background!50!white}
\setbeamercolor{block body alerted}{bg=alert!8!background,fg=main!30!black}
\setbeamercolor{block title example}{bg=example,fg=background!50!white}
\setbeamercolor{block body example}{bg=example!9!background,fg=main!30!black}

\setbeamercolor{normal text}{fg=black,bg=background}
\setbeamercolor{itemize item}{fg=ArduinoOrange}
\setbeamercolor{itemize subitem}{fg=ArduinoOrange}
\setbeamercolor{enumerate item}{fg=ArduinoOrange!90!black}
\setbeamercolor{enumerate subitem}{fg=ArduinoOrange!90!black}
\setbeamercolor{description item}{fg=ArduinoOrange}
\setbeamercolor{description subitem}{fg=ArduinoOrange}

\setbeamerfont{frametitle}{size=\large}
\setbeamerfont{footline}{size=\tiny}

\setbeamersize{text margin left=0.4cm,text margin right=0.4cm}


% PACOTES ==============================


\usepackage[brazilian]{babel}
\usepackage[type={CC},modifier={zero},version={1.0},lang=brazilian]{doclicense}
\usepackage[cache=false,draft,section]{minted} % XXX: remover draft no final
\usepackage{graphicx}
\usepackage{multirow}


% MINTED ===============================


\usemintedstyle{perldoc}
\usemintedstyle[arduino]{arduino}

\setminted{
%  beameroverlays=true,
	bgcolor=white,
	highlightcolor=ArduinoYellow!20!white,
	autogobble=true,
	tabsize=2,
	fontsize=\footnotesize,
	baselinestretch=1,
	breaklines=true,
	breakbytoken=true,
	escapeinside=@@,
	mathescape=true
}


% COMPORTAMENTO ========================


\mode<handout>{\AtBeginSubsection{}}
\AtBeginSection{}
\AtEndDocument{\FinalFrame}

% Diretórios de imagens comum e local, respectivamente.
\graphicspath{{../common/graphics/}{graphics/}}

% Aumenta o espaçamento entre linhas de tabelas em 10%.
\renewcommand{\arraystretch}{1.1}


% MACROS ===============================


% Formatação de e-mails
\newcommand{\Email}[1]{\href{mailto:#1}{\texttt{#1}}}

% Formatação de URLs; parâmetro opcional é o texto do hyperlink
\newcommand{\Link}[2][]{\href{#2}{\textcolor{ArduinoTeal}{\ifthenelse{\equal{#1}{}}{\texttt{#2}}{#1}}}}

% Preenchimento do endereço de circuitos de sketches
\newcommand{\CircuitPath}[1]{sketches/#1/#1.pdf} % Abstração do diretório local de sketches

% Inserir circuito de uma sketch
\newcommand{\Circuit}[2][]{\includegraphics[#1]{\CircuitPath{#2}}}

% Preenchimento do endereço de sketches
\newcommand{\SketchPath}[1]{{sketches/#1/#1.ino}} % Abstração do diretório local de sketches

% Inserir sketch e ignorar as linhas do identificador SPDX
\newcommand{\Sketch}[1]{\inputminted[firstline=3]{arduino}{\SketchPath{#1}}}

% Coloração de sintaxe de elementos especiais em código.
\newcommand{\Highlight}[1]{\textbf{\textcolor{ArduinoOrange}{#1}}}

% Coloração de sintaxe de valores iniciais de variáveis em código.
\newcommand{\HighlightInit}[1]{\textbf{\textcolor{ArduinoBrown}{#1}}}

% Coloração de sintaxe de tipos de variáveis em código.
\newcommand{\HighlightType}[1]{\texttt{\textcolor{ArduinoTeal}{#1}}}

% Último frame de cada apresentação predefinido
\newcommand{\FinalFrame}{%
\begin{frame}[focus,b]{Por enquanto é só}\centering
	\begin{columns}[c,onlytextwidth]
	\column{0.7\linewidth}
		Por enquanto é só.
	\column{0.3\linewidth}
		\begin{alertblock}{Material:}
			\medskip
			\href{https://github.com/guilhermgonzaga/MinicursoArduino}%
				{\includegraphics[width=\linewidth]{qrGithub.png}}
		\end{alertblock}
	\end{columns}
	\vfill
	{\scriptsize \doclicenseThis}
\end{frame}%
}


% METADADOS ============================


% O título será definido em cada slide, não aqui
\subtitle{Minicurso de Arduino}
\author{}
\institute[Facom/UFMS]{\vspace{-\bigskipamount}
	\begin{columns}[c,onlytextwidth]
		\column{0.6\linewidth}
			Faculdade de Computação -- Facom\bigskip\\
			Universidade Federal de Mato Grosso do Sul -- UFMS
		\column{0.22\linewidth}
			\includegraphics[width=\linewidth]{facomGraph.png}
	\end{columns}
}
\date{\the\year}
