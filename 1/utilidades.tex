\section{Utilidades do Arduino}


\subsection{\texttt{String}}


\begin{frame}[b]{\insertsection: \insertsubsection}

	A linguagem que usamos (C++) não possui suporte "nativo"\@ (\textit{built-in}) para strings. Portanto, em ambientes limitados, como microcontroladores, é necessário trabalhar de forma mais primitiva, usando \Link[cadeias de caracteres]{https://www.arduino.cc/reference/pt/language/variables/data-types/string}.

	\medskip
	Por esse motivo, criou-se a classe auxiliar \Highlight{\texttt{String}}, que facilita manipular texto dando suporte a operações comuns, como busca, comparação e concatenação.

	\vfill
	Documentação: \Link[\texttt{String}]{https://www.arduino.cc/reference/pt/language/variables/data-types/stringobject}

\end{frame}


\begin{frame}[fragile]{\insertsection: \insertsubsection}

	Há diversas formas de inicializar uma \texttt{String}.

	\begin{minted}{arduino}
		String s1 = "um texto";          // com cadeia de caracteres
		String s2 = s1 + " mais longo";  // concatenando duas strings
		String s3 = String('a');         // com char
	\end{minted}

	\begin{minted}{arduino}
		String s4 = String(42);      // com int/long; representação decimal
		String s5 = String(42, BIN); // com int/long; representação binária
		String s6 = String(42, HEX); // com int/long; representação hexadecimal
		String s7 = String(3.14, 2); // com float e precisão opcional
	\end{minted}

\end{frame}


\begin{frame}[fragile]{\insertsection: \insertsubsection}

	E várias operações a se fazer.

	\begin{minted}{arduino}
		s2.startsWith(s1);  // Retorna true
		s2.startsWith(s3);  // Retorna false
	\end{minted}

	\begin{minted}{arduino}
		s2.substring(3);     // Retorna "texto mais longo"
		s2.substring(3, 7);  // Retorna "text" (o final é excluído)
	\end{minted}

	\begin{minted}{arduino}
		s4.toInt();    // Retorna 42 (traduz apenas decimal)
		s7.toFloat();  // Retorna 3.14
	\end{minted}

	\begin{minted}{arduino}
		String verso = "a sapa na lava a pa";
		verso.replace('a', 'e');
		// Agora verso vale "e sepe ne leve e pe"
	\end{minted}

\end{frame}


\subsection{\texttt{Serial}}


\begin{frame}[b]{\insertsection: \insertsubsection}

	O \Link[\texttt{Serial}]{https://www.arduino.cc/reference/pt/language/functions/communication/serial} é um objeto especial criado para gerenciar comunicação assíncrona entre o micro e outros dispositivos. Nossas placas podem se comunicar com ele tanto por USB quanto pelos pinos \textbf{0}~e~\textbf{1}.

	\medskip
	Na verdade, existem outros três objetos similares no Mega 2560:
	\begin{description}
		\item[\texttt{Serial1}] usa pinos 19 e 18
		\item[\texttt{Serial2}] usa pinos 17 e 16
		\item[\texttt{Serial3}] usa pinos 15 e 14
	\end{description}

Estes podem ser usados da mesma forma que o original.

	\vfill
	Leitura interessante: \Link[Serial~Communication]{https://learn.sparkfun.com/tutorials/serial-communication/all}

\end{frame}


% TODO:
% \begin{frame}{\insertsection: \insertsubsection}

% 	\Link[begin]{https://www.arduino.cc/reference/pt/language/functions/communication/serial/begin}

% \end{frame}


% TODO:
% \begin{frame}{\insertsection: \insertsubsection}

% 	\Link[print(ln)]{https://www.arduino.cc/reference/pt/language/functions/communication/serial/print}

% \end{frame}


% TODO:
% \begin{frame}{\insertsection: \insertsubsection}

% 	\Link[available]{https://www.arduino.cc/reference/pt/language/functions/communication/serial/available}

% \end{frame}


% TODO:
% \begin{frame}{\insertsection: \insertsubsection}

% 	Funções de leitura:
% 	\begin{itemize}
% 		\item read
% 		\item readString(Until)
% 		\item parseInt, parseFloat
% 		\item find(Until) (?)
% 	\end{itemize}

% \end{frame}


% TODO:
% \begin{frame}{\insertsection: \insertsubsection}

% 	\Link[serialEvent]{https://www.arduino.cc/reference/pt/language/functions/communication/serial/serialevent}

% \end{frame}
