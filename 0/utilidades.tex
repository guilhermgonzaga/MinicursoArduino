% SPDX-License-Identifier: CC0-1.0

\section{Utilidades do Arduino}


\begin{frame}{\insertsection}

	A plataforma Arduino define várias utilidades, que vamos ver conforme necessário. A \Link[documentação]{https://www.arduino.cc/reference/pt} descreve todas.

\end{frame}


\subsection{\texttt{String}}


\begin{frame}[b]{\insertsection: \insertsubsection}

	A linguagem que usamos (C++) não possui suporte "nativo"\@ (\textit{built-in}) para strings. Portanto, em ambientes limitados, como microcontroladores, é necessário trabalhar de forma mais primitiva, usando \Link[cadeias de caracteres]{https://www.arduino.cc/reference/pt/language/variables/data-types/string}.

	\medskip
	Por esse motivo, criou-se a classe auxiliar \Highlight{\texttt{String}}, que facilita manipular texto dando suporte a operações comuns, como busca, comparação e concatenação.

	\vfill
	Documentação: \Link[\texttt{String}]{https://www.arduino.cc/reference/pt/language/variables/data-types/stringobject}

\end{frame}


\begin{frame}[fragile]{\insertsection: \insertsubsection}

	Há diversas formas de inicializar uma \texttt{String}.

	\begin{minted}{arduino}
		String s1 = "um texto";          // com cadeia de caracteres
		String s2 = s1 + " mais longo";  // concatenando duas strings
		String s3 = String('a');         // com char
	\end{minted}

	\begin{minted}{arduino}
		String s4 = String(3.14, 2); // com float e precisão opcional
		String s5 = String(42);      // com int/long; representação decimal
		String s6 = String(42, BIN); // com int/long; representação binária
		String s7 = String(42, HEX); // com int/long; representação hexadecimal
	\end{minted}

\end{frame}


\begin{frame}[fragile]{\insertsection: \insertsubsection}

	E várias operações a se fazer.

	\begin{minted}{arduino}
		s2.startsWith(s1);  // Retorna true
		s2.startsWith(s3);  // Retorna false
	\end{minted}

	\begin{minted}{arduino}
		s2.substring(3);     // Retorna "texto mais longo"
		s2.substring(3, 7);  // Retorna "text" (o final é excluído)
	\end{minted}

	\begin{minted}{arduino}
		s4.toFloat();  // Retorna 3.14
		s5.toInt();    // Retorna 42 (traduz apenas decimal)
	\end{minted}

	\begin{minted}{arduino}
		String verso = "a sapa na lava a pa";
		verso.replace('a', 'e');
		// Agora verso vale "e sepe ne leve e pe"
	\end{minted}

\end{frame}


\subsection{Constantes Úteis}


\begin{frame}{\insertsection: \insertsubsection}

	Há várias constantes definidas para auxiliar durante a programação.
	\begin{description}[\texttt{LED\_BUILTIN}] % Cópia do item mais longo
		\item[\texttt{EULER}] A constante $e$.
		\item[\texttt{PI}] A constante $\pi$.
		% \item[\texttt{TWO\_PI}] A constante $2\pi$.
		\item[\texttt{LOW}] Nível lógico baixo (0).
		\item[\texttt{HIGH}] Nível lógico alto (1).
		\item[\texttt{LED\_BUILTIN}] Pino conectado a um LED na placa. Para nós, o pino 13.
		\item[\texttt{...}]
	\end{description}

\end{frame}


\subsection{Funções Especiais}


\begin{frame}[fragile]{\insertsection: \insertsubsection}

	Todo programa para Arduino precisa definir estas duas funções.

	\begin{minted}{arduino}
		void setup() {

		}

		void loop() {

		}
	\end{minted}

	\begin{description}
		\item[\texttt{setup}] executa uma só vez, quando a placa reinicia.
		\item[\texttt{loop}] executa repetidamente após \texttt{\Highlight{setup}} retornar.
	\end{description}

\end{frame}


\begin{frame}[b,fragile]{\insertsection: \insertsubsection}

	Configura o pino especificado como entrada ou saída.
	\begin{minted}{arduino}
		void pinMode(byte pin, byte mode);
	\end{minted}

	O valor de \texttt{\textbf{mode}} deve ser um destes abaixo.
	\begin{description}
		\item[\texttt{OUTPUT}] modo para acionamento.
		\item[\texttt{INPUT}] modo para leitura.
%		\item[\texttt{INPUT\_PULLUP}] modo para leitura com resistor de pull-up interno.
	\end{description}

	\vfill
	Leitura interessante: \Link[Digital Pins]{https://www.arduino.cc/en/Tutorial/Foundations/DigitalPins}

\end{frame}


\begin{frame}[b,fragile]{\insertsection: \insertsubsection}

	Aplica um nível lógico ao pino especificado.
	\begin{minted}{arduino}
		void digitalWrite(byte pin, byte val);
	\end{minted}

	O valor de \texttt{\textbf{val}} deve ser um destes abaixo.
	\begin{description}
		\item[\texttt{HIGH}] nível lógico alto
		\item[\texttt{LOW}] nível lógico baixo
	\end{description}

	\vfill
	Leitura interessante: \Link[Circuito Digital]{https://pt.wikipedia.org/wiki/Circuito\_digital}

\end{frame}
