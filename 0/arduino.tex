\section{Arduino}


\begin{frame}{O Projeto Arduino}
  Surgiu em 2005 com base em outro projeto da época, chamado \textit{Wiring}, cuja proposta era uma plataforma eletrônica aberta e fácil para iniciantes.

  % TODO Inexpensive, Cross-platform, Simple and clear programming environment, Open source and extensible hardware/software
\end{frame}


\subsection{Hardware: Mega 2560}


\begin{frame}{\insertsubsection}
  Nossa placa usa o microcontrolador \textcolor{CustomTeal}{ATmega2560}
  \begin{description}
    \item[7 V -- 12 V] Tensão de alimentação recomendada
    \item[5 V] Tensão de operação
    \item[16 MHz] Frequência de operação (\textit{clock-rate})
    \item[256 kB] Memória flash
    \item[8 kB] SRAM
    \item[4 kB] EEPROM
    \item[54] Pinos digitais (\textbf{\textcolor{CustomOrange}{15}} com saída PWM)
    \item[16] Pinos com leitura analógica
    \item[20 mA] Limite de corrente por pino
  \end{description}
\end{frame}


{ \definecolor{background}{named}{white}
\begin{frame}{\insertsubsection}
  \includegraphics[width=\linewidth]{mega2560.jpg}
\end{frame}
}


\subsection{Software: IDE}


\begin{frame}[b]{\insertsubsection}
  O ambiente de desenvolvimento que usaremos pode ser baixado em
  \begin{center}
    \Link[\texttt{arduino.cc/en/main/software}]{https://www.arduino.cc/en/main/software}
  \end{center}

  \vfill
  No entanto, há outras ferramentas, como
  \begin{enumerate}
    \item \Link[XOD]{https://xod.io/}
    \item \Link[Tinkercad Circuits]{https://www.tinkercad.com/learn/circuits}
    \item \Link[Arduino Create]{https://create.arduino.cc/}
    \item \Link[Arduino Pro IDE]{https://github.com/arduino/arduino-pro-ide/releases/latest}
  \end{enumerate}
\end{frame}


{ \definecolor{background}{named}{white}

\begin{frame}{\insertsubsection}
  \begin{columns}[onlytextwidth]
  \column[c]{0.55\linewidth}
    \includegraphics[height=0.85\textheight]{ideMainScreen.png}
  \column[c]{0.45\linewidth}
    \begin{tabular}{p{2ex}p{0.85\linewidth}}
      \multirow{2}{*}{\includegraphics[height=3ex]{ideToolbarVerify.png}} & \textcolor{CustomTeal}{Verificar} \\
      \smallskip & {\small Verifica erros de compilação no seu código.} \\
      \multirow{2}{*}{\includegraphics[height=3ex]{ideToolbarUpload.png}} & \textcolor{CustomTeal}{Carregar} \\
      \smallskip & {\small Compila e carrega seu código para a placa.} \\
      \multirow{2}{*}{\includegraphics[height=3ex]{ideToolbarNew.png}} & \textcolor{CustomTeal}{Novo} \\
      \smallskip & {\small Cria um novo \textit{sketch}.} \\
      \multirow{2}{*}{\includegraphics[height=3ex]{ideToolbarOpen.png}} & \textcolor{CustomTeal}{Abrir} \\
      \smallskip & {\small Abre um \textit{sketch}.} \\
      \multirow{2}{*}{\includegraphics[height=3ex]{ideToolbarSave.png}} & \textcolor{CustomTeal}{Salvar} \\
      \smallskip & {\small Salva seu \textit{sketch}.} \\
      \multirow{2}{*}{\includegraphics[height=3ex]{ideToolbarSerialMonitor.png}} & \textcolor{CustomTeal}{Monitor Serial} \\
      \smallskip & {\small É um Terminal para interação com a placa.} \\
    \end{tabular}
  \end{columns}
\end{frame}


\begin{frame}{\insertsubsection}
  \includegraphics[height=0.85\textheight,trim={0 28mm 0 0}]{ideMenuToolsBoard.png}
\end{frame}


\begin{frame}{\insertsubsection}
  \includegraphics[height=0.85\textheight]{ideMenuExamplesBlink.png}
\end{frame}


\begin{frame}[focus]
  Pluguem suas placas.
\end{frame}


\begin{frame}{\insertsubsection}
  \includegraphics[height=0.85\textheight]{ideMenuToolsPort.png}
\end{frame}

} % Fim do \defnecolor
