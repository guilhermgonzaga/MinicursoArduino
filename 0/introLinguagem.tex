\section{Introdução à Linguagem}


\begin{frame}{\insertsection}
  Arduino usa a linguagem C++, mas qualquer um que conheça uma linguagem baseada em C terá facilidade ao programar. \medskip

  Vamos mostrar o básico para que vocês se virem. Contudo, recomendamos que pratiquem no tempo livre.
\end{frame}


\begin{frame}{\insertsection: variáveis e tipos}
  Variáveis são declaradas da seguinte forma.
  \inputminted[firstline=1,lastline=1]{arduino}{\SketchPath{introLinguagem}}
  Também é possível inicializar uma variável na declaração.
  \inputminted[firstline=2,lastline=2]{arduino}{\SketchPath{introLinguagem}}

  É preciso especificar um tipo na declaração da variável. Essa característica se chama tipagem estática.

  % TODO Note que todo comando termina com ponto e vírgula.
\end{frame}


\begin{frame}{\insertsection: tipos e variáveis}
  Estes tipos são suficientes para o minicurso.

  \begin{description}
    \item[\texttt{bool}] guarda \textbf{\texttt{true}} ou \textbf{\texttt{false}}; também chamado \textbf{\texttt{\textcolor{CustomOrange}{boolean}}}
    \item[\texttt{char}] inteiros de 8 bits
    \item[\texttt{byte}] inteiros não negativos de 8 bits
%    \item[\texttt{short}] inteiros de 16 bits
    \item[\texttt{int}] inteiros de 16 bits
%    \item[\texttt{word}] inteiros não negativos de 16 bits
    \item[\texttt{long}] inteiros de 32 bits
    \item[\texttt{float}] ponto flutuante de 32 bits
%    \item[\texttt{double}] ponto flutuante de 32 bits
    \item[\texttt{String}] texto
    \item[\texttt{...}] \textit{tipos específicos para cada módulo}
  \end{description}

  \vfill
  \textbf{Obs.:} esses tamanhos são referentes ao Mega 2560 e podem mudar em outras plataformas.
\end{frame}


\begin{frame}{\insertsection: tipos e modificadores}
  Variáveis \textbf{de inteiros} podem representar números não negativos se modificadas com \textbf{\texttt{\textcolor{CustomOrange}{unsigned}}}.
  \inputminted[firstline=4,lastline=7]{arduino}{\SketchPath{introLinguagem}}

  \pause
  Pensando bem, \textbf{\texttt{\textcolor{CustomOrange}{byte}}} é o mesmo que \textbf{\texttt{\textcolor{CustomOrange}{unsigned~char}}}.
\end{frame}


\begin{frame}{\insertsection: tipos e modificadores}
  \textcolor{red}{TODO} const
\end{frame}


\begin{frame}{\insertsection: operadores}
  \textcolor{red}{TODO}
\end{frame}


\begin{frame}{\insertsection: if-else}
  \textcolor{red}{TODO}
\end{frame}


\begin{frame}{\insertsection: while e for}
  \textcolor{red}{TODO}
\end{frame}


\begin{frame}{\insertsection: funções}
  \textcolor{red}{TODO}
\end{frame}
