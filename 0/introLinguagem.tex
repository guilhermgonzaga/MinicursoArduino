\section{Introdução à Linguagem}


\begin{frame}{\insertsection}
  \only<1>{%
    Arduino usa a linguagem C++, mas qualquer um que conheça uma linguagem baseada em C terá facilidade ao programar.

    \bigskip
    Vamos mostrar só o básico para que vocês se virem.%
  }

  \only<2>{%
    \bigskip
    Diferenças claras entre C++ e Python:
    \begin{enumerate}
    \item Todo comando termina com ponto e vírgula.
    \item Quebras de linha e espaços extras não fazem diferença. Portanto, lembre-se que há vários estilos para se usar. Só não faça algo ilegível ou inconsistente.
    \item É preciso especificar um tipo na declaração duma variável. Essa característica se chama tipagem estática.
    \end{enumerate}%
  }
\end{frame}


\subsection{Tipos e variáveis}


\begin{frame}{\insertsection: \insertsubsection}
  Variáveis são declaradas da seguinte forma.
  \inputminted[firstline=1,lastline=1]{arduino}{sketches/introLinguagem/.variaveis.ino}

  Também é possível inicializar uma variável na declaração.
  \inputminted[firstline=2,lastline=2]{arduino}{fsketches/introLinguagem/.variaveis.ino}
\end{frame}


\begin{frame}{\insertsection: \insertsubsection}
  Estes tipos são suficientes para o minicurso.

  \begin{description}
    \item[\texttt{bool}] guarda \textbf{\texttt{true}} ou \textbf{\texttt{false}}; também chamado \textbf{\texttt{\textcolor{CustomOrange}{boolean}}}
    \item[\texttt{char}] inteiros de 8 bits
    \item[\texttt{byte}] inteiros não negativos de 8 bits
%    \item[\texttt{short}] inteiros de 16 bits
    \item[\texttt{int}] inteiros de 16 bits
%    \item[\texttt{word}] inteiros não negativos de 16 bits
    \item[\texttt{long}] inteiros de 32 bits
    \item[\texttt{float}] ponto flutuante de 32 bits; equivale a \textbf{\texttt{\textcolor{CustomOrange}{double}}}
    \item[\texttt{String}] texto
    \item[\texttt{void}] tipo que significa "nada"; variáveis não podem ter esse tipo
    \item[\texttt{...}] \textit{tipos específicos para cada módulo}
  \end{description}

  \vfill
  \textbf{Obs.:} esses tamanhos são referentes ao Mega 2560 e \textbf{são diferentes} em outras plataformas.
\end{frame}


\begin{frame}{\insertsection: \insertsubsection}
  Variáveis \textbf{de inteiros} podem representar números não negativos se modificadas com \textbf{\texttt{\textcolor{CustomOrange}{unsigned}}}.
  \inputminted[firstline=4,lastline=8]{arduino}{sketches/introLinguagem/.variaveis.ino}

  \pause
  Respondam: que tipo é equivalente a \textbf{\texttt{\textcolor{CustomOrange}{unsigned~char}}}?\\
  \only<0|handout:2>{R: \textbf{\texttt{\textcolor{CustomOrange}{byte}}}}
\end{frame}


\begin{frame}{\insertsection: \insertsubsection}
  Quando não se quer que o valor duma variável mude, pode-se usar o modificador \textbf{\texttt{\textcolor{CustomOrange}{const}}}.

  \inputminted[firstline=10,lastline=14]{arduino}{sketches/introLinguagem/.variaveis.ino}

  Constantes devem ser inicializadas na declaração e não podem ter seu valor alterado.
\end{frame}


\begin{frame}{\insertsection: \insertsubsection}
  Se não entendeu direito, use o especificador \textbf{\texttt{\textcolor{CustomOrange}{auto}}} e deixe que o compilador adivinhe os tipos para você!
  \inputminted[firstline=16,lastline=20]{arduino}{sketches/introLinguagem/.variaveis.ino}

  \pause
  Só tenha cuidado com o tipo \textbf{\texttt{\textcolor{CustomOrange}{String}}}.
  \inputminted[firstline=22,lastline=26]{arduino}{sketches/introLinguagem/.variaveis.ino}
\end{frame}


\subsection{Operadores}


\begin{frame}{\insertsection: \insertsubsection}
  \begin{columns}
  \column[b]{0.51\linewidth}
    \begin{block}{\centering Alguns dos operadores de C++.\\{\small A precedência é decrescente por linha.}}
      \begin{tabular}{|c|l|}                                        \hline
        \textbf{Categoria} & \textbf{Operador}                   \\ \hline
        Pós-fixados        & \multirow{2}{*}{\texttt{++ {-}{-}}} \\
        (unários)          &                                     \\ \hline
        Prefixados         & \texttt{++ {-}{-} + -}              \\
        (unários)          & \texttt{(\textbf{\textcolor{CustomTeal}{tipo}}) !}          \\ \hline
        Multiplicativos\textcolor{CustomOrange}{\textbf{*}} & \texttt{* / \%} \\ \hline
        Aditivos           & \texttt{+ -}                        \\ \hline
        Deslocamento       & \texttt{{<}{<} {>}{>}}              \\ \hline
        Relacionais        & \texttt{< > <= >=}                  \\ \hline
        Igualdade          & \texttt{== !=}                      \\ \hline
        Lógicos            & \texttt{\&\& ||}                    \\ \hline
%        Condicionais       & \texttt{?:}                         \\ \hline
        \multirow{2}{*}{Atribuição\textcolor{CustomOrange}{\textbf{**}}} & \texttt{= *= /= \%= +=} \\
                           & \texttt{-= <{<}= >{>}=}             \\ \hline
      \end{tabular}
    \end{block}

  \column[b]{0.49\linewidth}
    \begin{itemize}
    \item Existem também \Link[representações~alternativas]{https://pt.cppreference.com/w/cpp/language/operator\_alternative}.\\
    \texttt{~!~$\Longleftrightarrow$~not}\\
    \texttt{\&\&~$\Longleftrightarrow$~and}\\
    \texttt{||~$\Longleftrightarrow$~or}\smallskip\\
    \item[\textbf{*}] O operador \textbf{\texttt{/}} faz divisão \textbf{truncada} em inteiros e \textbf{real} em ponto flutuante.
    \item[\textbf{**}] Uma atribuição do tipo \textbf{\texttt{a~?=~b}} é o mesmo que \textbf{\texttt{a~=~a~?~b}}.
    \end{itemize}
  \end{columns}
\end{frame}


\subsection{Condicionais}


\begin{frame}{\insertsection: \insertsubsection}
  Há várias formas de se usar a estrutura \textit{if-else}, com funcionamento igual a de outras linguagens.
  \begin{columns}[t]
  \column{0.28\linewidth}
    \inputminted[firstline=01,lastline=03]{arduino}{sketches/introLinguagem/.condicionais.ino}
  \column{0.28\linewidth}
    \inputminted[firstline=01,lastline=06]{arduino}{sketches/introLinguagem/.condicionais.ino}
  \column{0.44\linewidth}
    \inputminted[firstline=8,lastline=16]{arduino}{sketches/introLinguagem/.condicionais.ino}
  \end{columns}
  Note que não é a indentação que define o corpo do \textit{if}, mas sim as chaves.
\end{frame}


\begin{frame}{\insertsection: \insertsubsection}
  Lembra que quebras de linha e espaço extra não fazem diferença?
  \begin{columns}[t]
  \column{0.28\linewidth}
    \inputminted[firstline=01,lastline=06]{arduino}{sketches/introLinguagem/.condicionais.ino}\vspace{-\bigskipamount}\\
    \inputminted[firstline=18,lastline=22]{arduino}{sketches/introLinguagem/.condicionais.ino}
  \column{0.28\linewidth}
    \inputminted[firstline=24,lastline=31]{arduino}{sketches/introLinguagem/.condicionais.ino}
  \column{0.44\linewidth}
    \inputminted[firstline=33,lastline=39]{arduino}{sketches/introLinguagem/.condicionais.ino}
  \end{columns}
\end{frame}


\begin{frame}{\insertsection: \insertsubsection}
  Só não faça algo assim:
  \begin{columns}[t]
  \column{0.5\linewidth}
    \inputminted[firstline=41,lastline=41]{arduino}{sketches/introLinguagem/.condicionais.ino}\\
    \includegraphics[width=\linewidth]{memeJackieChan.png}
  \column{0.5\linewidth}
    \inputminted[firstline=43,lastline=52]{arduino}{sketches/introLinguagem/.condicionais.ino}
  \end{columns}
\end{frame}


\subsection{Laços}


\begin{frame}{\insertsection: \insertsubsection}
  \inputminted[firstline=1,lastline=3]{arduino}{sketches/introLinguagem/.while.ino}
\end{frame}


\begin{frame}{\insertsection: \insertsubsection}
  A estrutura de um laço \textit{for} é a seguinte.
  \inputminted[firstline=01,lastline=03]{arduino}{sketches/introLinguagem/.for.ino}

  O laço \textit{while} equivalente é assim:
  \inputminted[firstline=05,lastline=10]{arduino}{sketches/introLinguagem/.while.ino}

%  Normalmente a inicialização é uma atribuição a variável.
  A inicialização pode incluir ou não uma \textbf{declaração}.
\end{frame}


\begin{frame}{\insertsection: \insertsubsection}
  Se houver declaração, cria-se uma variável visível somente dentro do laço.
  \inputminted[firstline=05,lastline=07]{arduino}{sketches/introLinguagem/.for.ino}

  Se não houver, usa-se uma variável já instanciada (precisa existir).
  \inputminted[firstline=09,lastline=14]{arduino}{sketches/introLinguagem/.for.ino}
\end{frame}


\begin{frame}{\insertsection: \insertsubsection}
  Exemplos:
  \begin{columns}
  \column{0.5\linewidth}
    \inputminted[firstline=16,lastline=18]{arduino}{sketches/introLinguagem/.for.ino}
  \column{0.5\linewidth}
    \inputminted[firstline=20,lastline=22]{arduino}{sketches/introLinguagem/.for.ino}
  \end{columns}
%  \pause
  Respondam: como seria um \textit{for} iterando as potências de 2?
\end{frame}


\subsection{Funções}


\begin{frame}{\insertsection: \insertsubsection}
  \textcolor{red}{TODO}
\end{frame}
