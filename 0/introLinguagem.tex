\section{Introdução à Linguagem}


\begin{frame}{\insertsection}

	\only<1>{%
		Arduino é programado em C++, mas qualquer um que conheça uma linguagem baseada em C terá facilidade ao programar.

		\bigskip
		Vamos mostrar só o básico para que vocês se virem.%
	}

	\only<2>{%
		\bigskip
		Diferenças claras entre C++ e Python:
		\begin{enumerate}
			\item Todo comando termina com ponto e vírgula.
			\item Quebras de linha e espaços extras não fazem diferença. Portanto, lembre-se que há vários estilos para se usar. Só não faça algo ilegível ou inconsistente.
			\item É preciso especificar um tipo na declaração duma variável. Essa característica se chama tipagem estática.
		\end{enumerate}%
	}

\end{frame}


\subsection{Tipos e variáveis}


\begin{frame}[fragile]{\insertsection: \insertsubsection}

	Variáveis são declaradas da seguinte forma.
	\begin{minted}{arduino}
		@\HighlightType{tipo}@ identificador;
	\end{minted}

	Também é possível inicializar uma variável na declaração.
	\begin{minted}{arduino}
		@\HighlightType{tipo}@ identificador = @\HighlightInit{valor\_inicial}@;
	\end{minted}

\end{frame}


\begin{frame}{\insertsection: \insertsubsection}

	Estes tipos são suficientes para o minicurso.

	\begin{description}
		\item[\texttt{bool}] guarda \textbf{\texttt{true}} ou \textbf{\texttt{false}}; também chamado \texttt{\HighlightSpecial{boolean}}
		\item[\texttt{char}] inteiros de 8 bits
		\item[\texttt{byte}] inteiros não negativos de 8 bits
%		\item[\texttt{short}] inteiros de 16 bits
		\item[\texttt{int}] inteiros de 16 bits
%		\item[\texttt{word}] inteiros não negativos de 16 bits
		\item[\texttt{long}] inteiros de 32 bits
		\item[\texttt{float}] ponto flutuante de 32 bits; equivale a \texttt{\HighlightSpecial{double}}
		\item[\texttt{String}] texto
		\item[\texttt{void}] tipo que significa "nada"; variáveis não podem ser deste tipo
		\item[\texttt{...}] \emph{tipos específicos para cada módulo}
	\end{description}

	\vfill
	\textbf{Obs.:} essas larguras são referentes ao ATmega2560 e \textbf{são diferentes} em outras plataformas.

\end{frame}


\begin{frame}[fragile]{\insertsection: \insertsubsection}

	Variáveis \textbf{de inteiros} podem representar números não negativos se modificadas com \texttt{\HighlightSpecial{unsigned}}.
	\begin{minted}{arduino}
		unsigned int natural = 10;
		natural = -10;  @$\leftarrow$ \alert{NÃO PODE}@

		int inteiro = 10;
		inteiro = -10;  @$\leftarrow$ \alert{PODE SIM}@
	\end{minted}

	\pause
	\bigskip
	Respondam: que tipo é equivalente a \texttt{\HighlightSpecial{unsigned~char}}?\\
	\only<0|handout:2>{R: \texttt{\HighlightSpecial{byte}}}

\end{frame}


\begin{frame}[fragile]{\insertsection: \insertsubsection}

	Quando se quer tornar uma variável imutável, pode-se usar o modificador \texttt{\HighlightSpecial{const}}.
	\begin{minted}{arduino}
		const String senha = "Fantasma!";
		senha = "Essa é a senha!";  @$\leftarrow$ \alert{NÃO PODE}@

		const float pi = 3.1416;
		pi = 3.0;                   @$\leftarrow$ \alert{NÃO PODE}@
	\end{minted}

	Constantes devem ser inicializadas na declaração e não podem ter seu valor alterado.

\end{frame}


\subsection{Operadores}


\begin{frame}{\insertsection: \insertsubsection}

	\begin{columns}[b]
	\column{0.51\linewidth}
		\begin{block}{\centering Alguns dos operadores de C++.\\{\small A precedência é decrescente por linha.}}
			\begin{tabular}{cl}
				Pós-fixados  & \multirow{2}{*}{\texttt{++ {-}{-}}}        \\
				(unários)    &                                            \\ \hline
				Prefixados   & \texttt{++ {-}{-} + -}                     \\
				(unários)    & \texttt{(\textbf{\HighlightType{tipo}}) !} \\ \hline
				Multiplicativos\HighlightSpecial{*} & \texttt{* / \%}     \\ \hline
				Aditivos     & \texttt{+ -}                               \\ \hline
				Deslocamento & \texttt{{<}{<} {>}{>}}                     \\ \hline
				Relacionais  & \texttt{< > <= >=}                         \\ \hline
				Igualdade    & \texttt{== !=}                             \\ \hline
				Bit-a-bit    & \texttt{\& | $\wedge$}                     \\ \hline
				Lógicos      & \texttt{\&\& ||}                           \\ \hline
%				Condicionais & \texttt{?:}                                \\ \hline
				\multirow{2}{*}{Atribuição\HighlightSpecial{**}} & \texttt{= *= /= \%= +=} \\
										 & \texttt{-= <{<}= >{>}=}                    \\ %\hline
			\end{tabular}
		\end{block}

	\column{0.49\linewidth}
		\begin{itemize}
		\item Existem também \Link[representações~alternativas]{https://pt.cppreference.com/w/cpp/language/operator\_alternative}.\\
		\texttt{~!~$\Longleftrightarrow$~not}\\
		\texttt{\&\&~$\Longleftrightarrow$~and}\\
		\texttt{||~$\Longleftrightarrow$~or}\smallskip\\
		\item[\textbf{*}] O operador \textbf{\texttt{/}} faz divisão \textbf{truncada} de inteiros e \textbf{real} de ponto flutuante.
		\item[\textbf{**}] Uma atribuição do tipo \textbf{\texttt{a~?=~b}} é o mesmo que \textbf{\texttt{a~=~a~?~b}}.
		\end{itemize}
	\end{columns}

\end{frame}


\subsection{Condicionais}


\begin{frame}[fragile]{\insertsection: \insertsubsection}

	Há várias formas de se usar a estrutura \textit{if-else}, com funcionamento igual a de outras linguagens.
	\begin{columns}[t]
	\column{0.28\linewidth}
		\begin{minted}{arduino}
			if (@\HighlightSpecial{condição}@) {
				ações;
			}
		\end{minted}

	\column{0.28\linewidth}
		\begin{minted}{arduino}
			if (@\HighlightSpecial{condição}@) {
				ações;
			}
			else {
				ações;
			}
		\end{minted}

	\column{0.44\linewidth}
		\begin{minted}{arduino}
			if (@\HighlightSpecial{uma condição}@) {
				ações;
			}
			else if (@\HighlightSpecial{outra condição}@) {
				ações;
			}
			else {
				ações;
			}
		\end{minted}
	\end{columns}

\end{frame}


\begin{frame}[fragile]{\insertsection: \insertsubsection}

	Lembra que quebras de linha e espaço extra não fazem diferença?
	\begin{columns}[t]
	\column{0.28\linewidth}
		\begin{minted}{arduino}
			if (@\HighlightSpecial{condição}@)
			{
				ações;
			}
		\end{minted}

		\vspace{-\medskipamount}
		\begin{minted}{arduino}
			if (@\HighlightSpecial{condição}@) {
				ações;
			} else {
				ações;
			}
		\end{minted}

	\column{0.28\linewidth}
		\begin{minted}{arduino}
			if (@\HighlightSpecial{condição}@)
			{
				ações;
			}
			else
			{
				ações;
			}
		\end{minted}

	\column{0.44\linewidth}
		\begin{minted}{arduino}
			if (@\HighlightSpecial{condição}@) {
				ações;
			} else if (@\HighlightSpecial{outra condição}@) {
				ações;
			} else {
				ações;
			}
		\end{minted}
	\end{columns}
	Não é a indentação que define o corpo, mas sim as chaves.

\end{frame}


\begin{frame}[fragile]{\insertsection: \insertsubsection}

	Mas escreva algo legível e consistente:
	\begin{columns}[t]
	\column{0.5\linewidth}
		\begin{minted}{arduino}
			if(@\HighlightSpecial{condição}@){ações;}else{ações;}
		\end{minted}
		\includegraphics[width=\linewidth]{memeJackieChan.png}

	\column{0.5\linewidth}
		\begin{minted}{arduino}
			if(@\HighlightSpecial{condição}@) {
				ações;
			}
			else
			if (@\HighlightSpecial{outra condição}@)
			{
				ações;
			} else {
				ações;
			}
		\end{minted}
	\end{columns}

\end{frame}


\subsection{Funções}


\begin{frame}[fragile]{\insertsection: \insertsubsection}

	Funções são definidas da seguinte forma.
	\begin{minted}{arduino}
		@\HighlightType{tipo\_de\_retorno}@ nome(@\HighlightSpecial{parâmetros}@) {
			ações;
			return @\HighlightSpecial{resultado}@;
		}
	\end{minted}

	\begin{itemize}
		\item Parâmetros são variáveis declaradas separadas por vírgulas.
		\item Se não houver parâmetros, pode-se deixar os parênteses vazios ou escrever \texttt{\textbf{void}}, tanto faz.
		\item O tipo de retorno da função é o tipo do \HighlightSpecial{resultado} retornado com o comando \texttt{\textbf{return}}.
		\item Se a função não retorna nada, então seu tipo de retorno é \texttt{\textbf{void}} e pode-se omitir o comando \texttt{\textbf{return}}.
	\end{itemize}

\end{frame}


\begin{frame}[fragile]{\insertsection: \insertsubsection}

	Exemplos:

	\inputminted{arduino}{\SketchPath{bare minimum}}

	Todo programa para Arduino precisa definir estas duas funções.
	\begin{description}
		\item[\texttt{setup}] é executada uma só vez, logo no início.
		\item[\texttt{loop}] é executada repetidamente após \texttt{\HighlightSpecial{setup}} retornar.
	\end{description}

\end{frame}


\begin{frame}[fragile]{\insertsection: \insertsubsection}

	Exemplos (inventados):
	\begin{columns}[t]
	\column{0.5\linewidth}
		\begin{minted}{arduino}
			long maior(long a, long b) {
				if (a > b) {
					return a;
				}
				else {
					return b;
				}
			}
		\end{minted}

	\column{0.5\linewidth}
		\begin{minted}{arduino}
			float media(float a, float b) {
				return (a + b) / 2;
			}
		\end{minted}

		\vspace{-\medskipamount}
		\begin{minted}{arduino}
			void troca(int& a, int& b) {
				int aux = a;
				a = b;
				b = aux;
			}
		\end{minted}
	\end{columns}

\end{frame}


\begin{frame}[fragile]{\insertsection: \insertsubsection}

	Exemplos (inventados):
	\begin{minted}{arduino}
		/* Descrição: verifica se <usuario> e <senha> são válidos.
		 * Retorna:   true caso as credenciais sejam válidas;
		 *            false caso contrário.
		 */
		bool checa_credenciais(String usuario, String senha) {
			if (@\HighlightSpecial{usuario existe}@ && @\HighlightSpecial{senha correta para ele}@) {
				return true;
			}
			else {
				return false;
			}
		}
	\end{minted}

\end{frame}


\subsection{Utilidades do Arduino}


\begin{frame}{\insertsection: \insertsubsection}

	A plataforma Arduino define várias utilidades, que vamos ver conforme necessário. A \Link[documentação]{https://www.arduino.cc/reference/pt/} descreve todas.

\end{frame}


\begin{frame}[fragile]{\insertsection: \insertsubsection}

	Configura o pino especificado como entrada ou saída.
	\begin{minted}{arduino}
		void pinMode(byte pin, byte mode);
	\end{minted}

	O valor de \texttt{\textbf{mode}} deve ser um destes abaixo.
	\begin{description}
		\item[\texttt{OUTPUT}] modo para escrita.
		\item[\texttt{INPUT}] modo para leitura.
%		\item[\texttt{INPUT\_PULLUP}] modo para leitura com resistor de pull-up interno.
	\end{description}

\end{frame}


\begin{frame}[fragile]{\insertsection: \insertsubsection}

	Aplica um nível lógico ao pino especificado.
	\begin{minted}{arduino}
		void digitalWrite(byte pin, byte val);
	\end{minted}

	O valor de \texttt{\textbf{val}} deve ser um destes abaixo.
	\begin{description}
		\item[\texttt{HIGH}] nível lógico alto
		\item[\texttt{LOW}] nível lógico baixo
	\end{description}

\end{frame}


\begin{frame}{\insertsection: \insertsubsection}

	Também há várias constantes definidas para nos ajudar.
	\begin{description}
		\item[\texttt{PI}] A constante $\pi$.
		\item[\texttt{EULER}] A constante $e$.
		\item[\texttt{LED\_BUILTIN}] Pino digital com um LED conectado. Para nós, o pino 13.
		\item[\texttt{...}] \emph{Muitas outras}.
	\end{description}

\end{frame}


\subsection{Laços}


\begin{frame}[fragile]{\insertsection: \insertsubsection}

	A estrutura \textit{while} funciona como a de outras linguagens: ações dentro do laço são repetidas enquanto a condição for verdadeira.
	\begin{minted}{arduino}
		while (@\HighlightSpecial{condição}@) {
			ações;
		}
	\end{minted}

\end{frame}


\begin{frame}[fragile]{\insertsection: \insertsubsection}

	A estrutura de um laço \textit{for} é esta:
	\begin{minted}{arduino}
		for (@\HighlightSpecial{inicialização}@; @\HighlightSpecial{condição}@; @\HighlightSpecial{passo}@) {
			ações;
		}
	\end{minted}
	e equivale a este \textit{while}:
	\begin{minted}{arduino}
		@\HighlightSpecial{inicialização}@;

		while (@\HighlightSpecial{condição}@) {
			ações;
			@\HighlightSpecial{passo}@;
		}
	\end{minted}

	A inicialização pode incluir ou não uma \textbf{declaração} de variável.

\end{frame}


\begin{frame}[fragile]{\insertsection: \insertsubsection}

	Se houver declaração, cria-se uma variável visível somente dentro do laço.
	\begin{minted}{arduino}
		for (@\HighlightType{tipo}@ nome = @\HighlightInit{valor\_inicial}@; @\HighlightSpecial{condição}@; @\HighlightSpecial{passo}@) {
			ações;
		}
	\end{minted}

	Se não houver, usa-se uma variável já declarada (precisa existir).
	\begin{minted}{arduino}
		@\HighlightType{tipo}@ nome;

		// ...

		for (nome = @\HighlightInit{valor\_inicial}@; @\HighlightSpecial{condição}@; @\HighlightSpecial{passo}@) {
			ações;
		}
	\end{minted}

\end{frame}


\begin{frame}[fragile]{\insertsection: \insertsubsection}

	Exemplos:
	\begin{columns}
	\column{0.5\linewidth}
		\begin{minted}{arduino}
			for (int i = 0; i < n; i++) {
				ações;
			}
		\end{minted}
		\begin{minted}{arduino}
			for (int i = n; i > 0; i--) {
				ações;
			}
		\end{minted}

	\column{0.5\linewidth}
		\begin{minted}{arduino}
			for (int i = 0; i < n; i += 2) {
				ações;
			}
		\end{minted}
		\begin{minted}{arduino}
			for (int i = 1; i < n; i += 2) {
				ações;
			}
		\end{minted}
	\end{columns}

%	\pause
	Respondam: como seria um \textit{for} iterando sobre as potências de 2?
%	\only<0|handout:2>{%
%		R:\\
%		\begin{minted}{arduino}
%		\end{minted}%
%	}

\end{frame}
