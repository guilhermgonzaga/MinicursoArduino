\section{Introdução à Linguagem}


\begin{frame}{\insertsection}
  Arduino usa a linguagem C++, mas qualquer um que conheça uma linguagem baseada em C terá facilidade ao programar. \medskip

  Vamos mostrar o básico para que vocês se virem. Contudo, recomendamos que pratiquem no tempo livre.

  % TODO Notem que todo comando termina com ponto e vírgula.
\end{frame}


\subsection{Tipos e variáveis}


\begin{frame}[b]{\insertsection: \insertsubsection}
  Variáveis são declaradas da seguinte forma.
  \inputminted[firstline=1,lastline=1]{arduino}{sketches/introLinguagem/variaveis.ino}

  Também é possível inicializar uma variável na declaração.
  \inputminted[firstline=2,lastline=2]{arduino}{fsketches/introLinguagem/variaveis.ino}

  \vfill
  \textbf{Obs.:} é preciso especificar um tipo na declaração duma variável. Essa característica se chama tipagem estática.\smallskip\\
\end{frame}


\begin{frame}{\insertsection: \insertsubsection}
  Estes tipos são suficientes para o minicurso.

  \begin{description}
    \item[\texttt{bool}] guarda \textbf{\texttt{true}} ou \textbf{\texttt{false}}; também chamado \textbf{\texttt{\textcolor{CustomOrange}{boolean}}}
    \item[\texttt{char}] inteiros de 8 bits
    \item[\texttt{byte}] inteiros não negativos de 8 bits
%    \item[\texttt{short}] inteiros de 16 bits
    \item[\texttt{int}] inteiros de 16 bits
%    \item[\texttt{word}] inteiros não negativos de 16 bits
    \item[\texttt{long}] inteiros de 32 bits
    \item[\texttt{float}] ponto flutuante de 32 bits; equivale a \textbf{\texttt{\textcolor{CustomOrange}{double}}}
    \item[\texttt{String}] texto
    \item[\texttt{...}] \textit{tipos específicos para cada módulo}
  \end{description}

  \vfill
  \textbf{Obs.:} esses tamanhos são referentes ao Mega 2560 e podem mudar em outras plataformas.
\end{frame}


\begin{frame}{\insertsection: \insertsubsection}
  Variáveis \textbf{de inteiros} podem representar números não negativos se modificadas com \textbf{\texttt{\textcolor{CustomOrange}{unsigned}}}.
  \inputminted[firstline=4,lastline=8]{arduino}{sketches/introLinguagem/variaveis.ino}

  \pause
  Respondam: que tipo é equivalente a \textbf{\texttt{\textcolor{CustomOrange}{unsigned~char}}}?\\
  \only<0|handout:2>{R: \textbf{\texttt{\textcolor{CustomOrange}{byte}}}}
\end{frame}


\begin{frame}{\insertsection: \insertsubsection}
  Quando não se quer que o valor duma variável mude, pode-se usar o modificador \textbf{\texttt{\textcolor{CustomOrange}{const}}}.

  \inputminted[firstline=10,lastline=11]{arduino}{sketches/introLinguagem/variaveis.ino}

  Constantes devem ser inicializadas na declaração e não podem ter seu valor alterado.
\end{frame}


\begin{frame}{\insertsection: \insertsubsection}
  Se não entendeu direito, use o especificador \textbf{\texttt{\textcolor{CustomOrange}{auto}}} e deixe que o compilador adivinhe os tipos para você!
  \inputminted[firstline=13,lastline=17]{arduino}{sketches/introLinguagem/variaveis.ino}

  \pause
  Mas cuidado com o tipo \textbf{\texttt{\textcolor{CustomOrange}{String}}}.
  \inputminted[firstline=19,lastline=23]{arduino}{sketches/introLinguagem/variaveis.ino}
\end{frame}


\subsection{Operadores}


\begin{frame}{\insertsection: \insertsubsection}
  \textcolor{red}{TODO}
\end{frame}


\subsection{Condicionais}


\begin{frame}{\insertsection: \insertsubsection}
  \textcolor{red}{TODO}
\end{frame}


\subsection{Laços}


\begin{frame}{\insertsection: \insertsubsection}
  \textcolor{red}{TODO}
\end{frame}


\subsection{Funções}


\begin{frame}{\insertsection: \insertsubsection}
  \textcolor{red}{TODO}
\end{frame}
