% SPDX-License-Identifier: CC0-1.0

\section{Introdução à Linguagem}


\begin{frame}[b]{\insertsection}

	Arduino é programado em C++, mas quem conhecer qualquer linguagem baseada em C terá facilidade. Veremos só o básico para que vocês se virem.

	\bigskip
	Características da linguagem C e suas derivadas:
	\begin{itemize}
		\item Todo comando termina com ponto e vírgula.
		\item Quebras de linha e espaços extras não fazem diferença. Portanto, há diversas formas de estilizar código.
		\item É preciso especificar tipos na declaração de variáveis. Essa característica se chama tipagem estática.
	\end{itemize}

	\vfill
	Leitura interessante: \Link[Tipagem fraca, forte, dinâmica e estática]{https://dev.to/joaoava/tipagem-fraca-forte-dinamica-e-estatica-g8k}

\end{frame}


\subsection{Variáveis e tipos}


\begin{frame}[fragile]{\insertsection: \insertsubsection}

	Variáveis são declaradas da seguinte forma.
	\begin{minted}{arduino}
		@\HighlightType{tipo}@ nome;
	\end{minted}

	Também é possível inicializar uma variável na declaração.
	\begin{minted}{arduino}
		@\HighlightType{tipo}@ nome = @\HighlightInit{valor~inicial}@;
	\end{minted}

\end{frame}


\begin{frame}{\insertsection: \insertsubsection}

	Estes tipos são suficientes para nós. Veja todos na \Link[documentação]{https://www.arduino.cc/reference/pt}.

	\begin{description}
		\item[\texttt{bool}] guarda \textbf{\texttt{true}} ou \textbf{\texttt{false}}%; também chamado \texttt{\Highlight{boolean}}
		\item[\texttt{char}] inteiro de 8 bits para representar caracteres
		\item[\texttt{byte}] inteiro não negativo de 8 bits\Highlight{**}
%		\item[\texttt{short}] inteiro de 16 bits\Highlight{*}
		\item[\texttt{int}] inteiro de 16 bits\Highlight{*}
		\item[\texttt{long}] inteiro de 32 bits\Highlight{*}
		\item[\texttt{float}] ponto flutuante de 32 bits
%		\item[\texttt{double}] ponto flutuante de 32 bits\Highlight{*}
		\item[\texttt{String}] texto\Highlight{**}
		\item[\texttt{void}] tipo que significa \emph{vazio}; tem usos bem específicos
		\item[\texttt{...}] \emph{tipos dedicados a cada sensor/módulo}
	\end{description}

	\vfill
	\begin{itemize}
		\item[\textbf{*}] São os tamanhos no ATmega2560 e \textbf{podem mudar} em outros micros.
		\item[\textbf{**}] Estes tipos só existem no ambiente Arduino.
	\end{itemize}

\end{frame}


\begin{frame}[fragile]{\insertsection: \insertsubsection}

	Variáveis \textbf{de inteiros} podem representar apenas números não negativos se modificadas com \texttt{\Highlight{unsigned}}.
	\begin{minted}{arduino}
		unsigned int natural = 10;
		natural = -10;  @$\leftarrow$ \alert{NÃO PODE}@

		int inteiro = 10;
		inteiro = -10;  @$\leftarrow$ \alert{PODE}@
	\end{minted}

\end{frame}


\begin{frame}[fragile]{\insertsection: \insertsubsection}

	Quando se quer tornar uma variável imutável, pode-se usar o modificador \texttt{\Highlight{const}}.
	\begin{minted}{arduino}
		const String senha = "Fantasma!";
		senha = "Essa é a senha!";  @$\leftarrow$ \alert{NÃO PODE}@

		const float pi;             @$\leftarrow$ \alert{NÃO PODE}@
		pi = 3.1416;                @$\leftarrow$ \alert{NÃO PODE}@
	\end{minted}

	Constantes devem ser inicializadas na declaração e não podem ter seu valor alterado.

\end{frame}


\subsection{Operadores}


\begin{frame}{\insertsection: \insertsubsection}

	\begin{columns}[b]
	\column{0.51\linewidth}
		\begin{block}{\centering Alguns dos operadores de C++\\{\small Precedência decrescente}}
			% Note as sequências '\ ' entre operadores para aumentar o espaçamento
			\begin{tabular}{cl}
				Unários      & \multirow{2}{*}{\texttt{++ \ {-}{-} \ [] \ .}} \\
				pós-fixados  &                                                \\ \hline
				Unários      & \texttt{++ \ {-}{-} \ + \ -}                   \\
				prefixados   & \texttt{(\textbf{\HighlightType{tipo}}) \ !}   \\ \hline
				Multiplicativos\Highlight{*} & \texttt{* \ / \ \%}            \\ \hline
				Aditivos     & \texttt{+ \ -}                                 \\ \hline
				Deslocamento & \texttt{{<}{<} \ {>}{>}}                       \\ \hline
				Relacionais  & \texttt{< \ > \ <= \ >=}                       \\ \hline
				Igualdade    & \texttt{== \ !=}                               \\ \hline
%				Bit-a-bit    & \texttt{\& \ $\wedge$ \ |}                     \\ \hline
				Lógicos      & \texttt{\&\& \ ||}                             \\ \hline
%				Condicionais & \texttt{?:}                                    \\ \hline
				\multirow{2}{*}{Atribuição\Highlight{**}} & \texttt{= \ *= \ /= \ \%=} \\
				             & \texttt{+= -= <{<}= >{>}=}                     \\ %\hline
			\end{tabular}
		\end{block}

	\column{0.49\linewidth}
		\begin{itemize}
%			\item[] Também há \Link[representações alternativas]{https://pt.cppreference.com/w/cpp/language/operator\_alternative} para alguns.\\
%			\texttt{~!~$\Longleftrightarrow$~not}\\
%			\texttt{\&\&~$\Longleftrightarrow$~and}\\
%			\texttt{||~$\Longleftrightarrow$~or}\smallskip\\
			\item[\textbf{*}] O operador \textbf{\texttt{/}} faz divisão \textbf{truncada} de inteiros e \textbf{real} de ponto flutuante.
			\item[\textbf{**}] Uma atribuição do tipo \textbf{\texttt{a~\Highlight{?}=~b}} é o mesmo que \textbf{\texttt{a~=~a~\Highlight{?}~b}}.
		\end{itemize}
	\end{columns}

\end{frame}

% TODO
%\begin{frame}[fragile]{\insertsection: \insertsubsection}
%	Exemplos:
%	\begin{columns}
%	\column{0.33\linewidth}
%
%		\begin{minted}{arduino}
%			int a = 5, b = 2;
%			a = a + b;  // Mesmo que a += b;
%		\end{minted}
%
%	\column{0.33\linewidth}
%		\mintinline{arduino}{bool funnyCauseItsTrue = !false;}
%		\begin{minted}{arduino}
%			bool programmerJoke = !false;
%
%			// It's funny because it's true.
%		\end{minted}
%
%	\column{0.33\linewidth}
%		\begin{minted}{arduino}
%
%		\end{minted}
%	\end{columns}
%
%\end{frame}


\subsection{Condicionais}


\begin{frame}[fragile]{\insertsection: \insertsubsection}

	Há várias formas de se usar a estrutura \emph{if-else}, com funcionamento igual a de outras linguagens.
	\begin{columns}[t]
	\column{0.28\linewidth}
		\begin{minted}{arduino}
			if (@\Highlight{condição}@) {
				ações;
			}
		\end{minted}

	\column{0.28\linewidth}
		\begin{minted}{arduino}
			if (@\Highlight{condição}@) {
				ações;
			}
			else {
				ações;
			}
		\end{minted}

	\column{0.44\linewidth}
		\begin{minted}{arduino}
			if (@\Highlight{uma condição}@) {
				ações;
			}
			else if (@\Highlight{outra condição}@) {
				ações;
			}
			else {
				ações;
			}
		\end{minted}
	\end{columns}

\end{frame}


\begin{frame}[fragile]{\insertsection: \insertsubsection}

	Lembra que quebras de linha e espaço extra não fazem diferença?
	\begin{columns}[t]
	\column{0.28\linewidth}
		\begin{minted}{arduino}
			if (@\Highlight{condição}@)
			{
				ações;
			}
		\end{minted}

		\vspace{-\medskipamount}
		\begin{minted}{arduino}
			if (@\Highlight{condição}@) {
				ações;
			} else {
				ações;
			}
		\end{minted}

	\column{0.28\linewidth}
		\begin{minted}{arduino}
			if (@\Highlight{condição}@)
			{
				ações;
			}
			else
			{
				ações;
			}
		\end{minted}

	\column{0.44\linewidth}
		\begin{minted}{arduino}
			if (@\Highlight{condição}@) {
				ações;
			} else if (@\Highlight{outra condição}@) {
				ações;
			} else {
				ações;
			}
		\end{minted}
	\end{columns}
	Não é a indentação que define o corpo, mas sim as chaves.

\end{frame}


\begin{frame}[fragile]{\insertsection: \insertsubsection}

	Mas escreva algo convencional!
	\begin{columns}[T]
	\column{0.5\linewidth}
	\centering
		\includegraphics[width=\linewidth]{memeJackieChan.png}
		{\small (como programador, é bom médico)}

	\column{0.5\linewidth}
		\begin{minted}{arduino}
			if(@\Highlight{condição}@) {
				ações;
			}
			else
			if (@\Highlight{outra condição}@)
			{
				ações;
			} else { ações; }
		\end{minted}
	\end{columns}

\end{frame}


\subsection{Funções}


\begin{frame}[fragile]{\insertsection: \insertsubsection}

	Funções são definidas da seguinte forma.
	\begin{minted}{arduino}
		@\HighlightType{tipo\_de\_retorno}@ nome(@\Highlight{parâmetros}@) {
			ações;
			return @\Highlight{resultado}@;
		}
	\end{minted}

	\begin{itemize}
		\item O tipo de retorno da função é o tipo do \Highlight{resultado} retornado com o comando \texttt{\textbf{return}}.
		\item Funções que não retornam valores têm tipo de retorno \HighlightType{void} e não precisam de \texttt{\textbf{return}} no final.
		\item Parâmetros são variáveis/referências passados para a função. Referências permitem mudar variáveis do código que chama a função.
	\end{itemize}

\end{frame}


\begin{frame}[b,fragile]{\insertsection: \insertsubsection}

	Exemplos:
	\begin{columns}[t]
	\column{0.5\linewidth}
		\begin{minted}{arduino}
			long maior(long a, long b) {
				if (a > b) {
					return a;
				}
				else {
					return b;
				}
			}
		\end{minted}

	\column{0.5\linewidth}
		\begin{minted}{arduino}
			void troca(int& a, int& b) {
				int aux = a;
				a = b;
				b = aux;
			}
		\end{minted}

		\vspace{-\medskipamount}
		\begin{minted}{arduino}
			float media(float a, float b) {
				return (a + b) / 2;
			}
		\end{minted}
	\end{columns}

	\vfill
	Leitura interessante: \Link[What is a reference?]{https://isocpp.org/wiki/faq/references\#overview-refs}

\end{frame}


\begin{frame}[fragile]{\insertsection: \insertsubsection}

	Exemplos:
	\begin{minted}{arduino}
		/* Descrição: verifica se <usuario> e <senha> são válidos.
		 * Retorna:   true caso as credenciais sejam válidas;
		 *            false caso contrário.
		 */
		bool checa_credenciais(String& usuario, String& senha) {
			if (@\Highlight{usuario existe}@ && @\Highlight{senha correta para ele}@) {
				return true;
			}
			else {
				return false;
			}
		}
	\end{minted}

\end{frame}


\subsection{Laços}


\begin{frame}[fragile]{\insertsection: \insertsubsection}

	A estrutura \emph{while} é similar em muitas linguagens: ações dentro do laço são repetidas enquanto a condição for verdadeira.
	\begin{minted}{arduino}
		while (@\Highlight{condição}@) {
			ações;
		}
	\end{minted}

\end{frame}


\begin{frame}[fragile]{\insertsection: \insertsubsection}

	A estrutura de um laço \emph{for} é esta:
	\begin{minted}{arduino}
		for (@\Highlight{inicialização}@; @\Highlight{condição}@; @\Highlight{passo}@) {
			ações;
		}
	\end{minted}
	e equivale a este \emph{while}:
	\begin{minted}{arduino}
		@\Highlight{inicialização}@;

		while (@\Highlight{condição}@) {
			ações;
			@\Highlight{passo}@;
		}
	\end{minted}

	A \Highlight{inicialização} pode incluir uma declaração de variável.

\end{frame}


\begin{frame}[b,fragile]{\insertsection: \insertsubsection}

	Se houver declaração, cria-se uma variável visível somente dentro do laço.
	\begin{minted}{arduino}
		for (@\HighlightType{tipo}@ nome = @\HighlightInit{valor~inicial}@; @\Highlight{condição}@; @\Highlight{passo}@) {
			ações;
		}
	\end{minted}

	Se não houver, usa-se uma variável já declarada.
	\begin{minted}{arduino}
		@\HighlightType{tipo}@ nome;

		// ...

		for (nome = @\HighlightInit{valor~inicial}@; @\Highlight{condição}@; @\Highlight{passo}@) {
			ações;
		}
	\end{minted}

	\vfill
	Leitura interessante: \Link[Escopo]{https://www.arduino.cc/reference/pt/language/variables/variable-scope-qualifiers/scope}

\end{frame}


\begin{frame}[fragile]{\insertsection: \insertsubsection}

	Exemplos:
	\begin{columns}
	\column{0.5\linewidth}
		\begin{minted}{arduino}
			for (int i = 0; i < n; i++) {
				ações;
			}
		\end{minted}
		\begin{minted}{arduino}
			for (int i = n-1; i >= 0; i--) {
				ações;
			}
		\end{minted}

	\column{0.5\linewidth}
		\begin{minted}{arduino}
			for (int i = 0; i < n; i += 2) {
				ações;
			}
		\end{minted}
		\begin{minted}{arduino}
			for (char c = 'a'; c < 'z'; c++) {
				ações;
			}
		\end{minted}
	\end{columns}

	\vfill
	Respondam: como seria um \emph{for} iterando sobre as potências de 2?

\end{frame}


\subsection{Vetores}


\begin{frame}[fragile]{\insertsection: \insertsubsection}

	Vetores são conjuntos de variáveis do mesmo tipo armazenadas em sequência.

	\bigskip
	Pode-se declarar a quantidade de elementos ou deixar que seja inferida a partir do inicializador.

	\begin{minted}{arduino}
		@\HighlightType{tipo}@ nome[@\Highlight{quantidade}@];
	\end{minted}

	\begin{minted}{arduino}
		@\HighlightType{tipo}@ nome[@\Highlight{quantidade}@] = {@\HighlightInit{valores~iniciais}@};
	\end{minted}

	\begin{minted}{arduino}
		@\HighlightType{tipo}@ nome[] = {@\HighlightInit{valores~iniciais}@};
	\end{minted}

	De toda forma, o tamanho de vetores é fixo e deve ser conhecido durante a compilação.

\end{frame}


\begin{frame}[fragile]{\insertsection: \insertsubsection}

	Exemplos:
	\begin{minted}{arduino}
		int pinos[] = {11, 12, 9};
		pinos[2] = 13;
		// Agora o vetor é {11, 12, 13}
	\end{minted}

	\begin{minted}{arduino}
		const int numMedidas = 4;
		float medidas[numMedidas];
		float media = 0;

		// [...] preenchendo
		// Agora o vetor é {20.5, 18.9, 17, 19.25}

		for (int i = 0; i < numMedidas; i += 1) {
			media = media + medidas[i];
		}
		media = media / numMedidas;
	\end{minted}

\end{frame}


\subsection{Utilidades do Arduino}


\begin{frame}{\insertsection: \insertsubsection}

	A plataforma Arduino define várias utilidades, que vamos ver conforme necessário. A \Link[documentação]{https://www.arduino.cc/reference/pt} descreve todas.

\end{frame}


\begin{frame}[fragile]{\insertsection: \insertsubsection}

	Todo programa para Arduino precisa definir estas duas funções.

	\Sketch{bare minimum}

	\begin{description}
		\item[\texttt{setup}] executa uma só vez, quando a placa reinicia.
		\item[\texttt{loop}] executa repetidamente após \texttt{\Highlight{setup}} retornar.
	\end{description}

\end{frame}


\begin{frame}[b,fragile]{\insertsection: \insertsubsection}

	Configura o pino especificado como entrada ou saída.
	\begin{minted}{arduino}
		void pinMode(byte pin, byte mode);
	\end{minted}

	O valor de \texttt{\textbf{mode}} deve ser um destes abaixo.
	\begin{description}
		\item[\texttt{OUTPUT}] modo para acionamento.
		\item[\texttt{INPUT}] modo para leitura.
%		\item[\texttt{INPUT\_PULLUP}] modo para leitura com resistor de pull-up interno.
	\end{description}

	\vfill
	Leitura interessante: \Link[Digital Pins]{https://www.arduino.cc/en/Tutorial/Foundations/DigitalPins}

\end{frame}


\begin{frame}[b,fragile]{\insertsection: \insertsubsection}

	Aplica um nível lógico ao pino especificado.
	\begin{minted}{arduino}
		void digitalWrite(byte pin, byte val);
	\end{minted}

	O valor de \texttt{\textbf{val}} deve ser um destes abaixo.
	\begin{description}
		\item[\texttt{HIGH}] nível lógico alto
		\item[\texttt{LOW}] nível lógico baixo
	\end{description}

	\vfill
	Leitura interessante: \Link[Circuito Digital]{https://pt.wikipedia.org/wiki/Circuito\_digital}

\end{frame}


\begin{frame}{\insertsection: \insertsubsection}

	Também há várias constantes definidas para nos ajudar.
	\begin{description}[\texttt{LED\_BUILTIN}] % Cópia do item mais longo
		\item[\texttt{EULER}] A constante $e$.
		\item[\texttt{PI}] A constante $\pi$.
		\item[\texttt{TWO\_PI}] A constante $2\pi$.
		\item[\texttt{LED\_BUILTIN}] Pino conectado a um LED na placa. Para nós, o pino 13.
		\item[\texttt{...}]
	\end{description}

\end{frame}
