\section{Utilidades do Arduino}


\subsection{\texttt{Serial}}


\begin{frame}[b]{\insertsection: \insertsubsection}

	O \Link[\texttt{Serial}]{https://www.arduino.cc/reference/pt/language/functions/communication/serial} é um objeto especial criado para gerenciar comunicação assíncrona entre o micro e outros dispositivos. Nossas placas podem se comunicar com ele tanto por USB quanto pelos pinos \textbf{0}~e~\textbf{1}.

	\medskip
	Na verdade, existem outros três objetos similares no Mega 2560:
	\begin{description}
		\item[\texttt{Serial1}] usa pinos 19 e 18
		\item[\texttt{Serial2}] usa pinos 17 e 16
		\item[\texttt{Serial3}] usa pinos 15 e 14
	\end{description}

Estes podem ser usados da mesma forma que o original.

	\vfill
	Leitura interessante: \Link[Serial~Communication]{https://learn.sparkfun.com/tutorials/serial-communication/all}

\end{frame}


% TODO
% \begin{frame}{\insertsection: \insertsubsection}

% 	\Link[begin]{https://www.arduino.cc/reference/pt/language/functions/communication/serial/begin}

% \end{frame}


% TODO
% \begin{frame}{\insertsection: \insertsubsection}

% 	\Link[print(ln)]{https://www.arduino.cc/reference/pt/language/functions/communication/serial/print}

% \end{frame}


% TODO
% \begin{frame}{\insertsection: \insertsubsection}

% 	\Link[available]{https://www.arduino.cc/reference/pt/language/functions/communication/serial/available}

% \end{frame}


% TODO
% \begin{frame}{\insertsection: \insertsubsection}

% 	Funções de leitura:
% 	\begin{itemize}
% 		\item read
% 		\item readString(Until)
% 		\item parseInt, parseFloat
% 		\item find(Until) (?)
% 	\end{itemize}

% \end{frame}


% TODO
% \begin{frame}{\insertsection: \insertsubsection}

% 	\Link[serialEvent]{https://www.arduino.cc/reference/pt/language/functions/communication/serial/serialevent}

% \end{frame}
