\documentclass[smaller]{beamer} % handout, unknownkeysallowed


% PALETA DE CORES ======================


\definecolor{CustomBrown}{HTML}{8C7965}
\definecolor{CustomOrange}{HTML}{E47128}
\definecolor{CustomTeal}{HTML}{00878F}
\definecolor{CustomYellow}{HTML}{E5AD24}
\definecolor{CustomLightBrown}{HTML}{BCAC9A}
\definecolor{CustomLightGrey}{HTML}{ECF1F1}
\definecolor{CustomLightOrange}{HTML}{F29C2D}
\definecolor{CustomLightTeal}{HTML}{62AEB2}


% TEMA =================================


\usetheme{focus}

\definecolor{main}{named}{CustomTeal}
\definecolor{background}{named}{CustomLightGrey}
\definecolor{alert}{named}{CustomLightOrange}
\definecolor{example}{named}{CustomBrown}

\setbeamercolor{block title}{bg=main!85!background, fg=background!50!white}
\setbeamercolor{block body}{bg=main!8!background, fg=main!20!black}
\setbeamercolor{block title alerted}{bg=alert, fg=background!50!white}
\setbeamercolor{block body alerted}{bg=alert!8!background, fg=main!30!black}
\setbeamercolor{block title example}{bg=example, fg=background!50!white}
\setbeamercolor{block body example}{bg=example!9!background, fg=main!30!black}

\setbeamercolor{normal text}{fg=black, bg=background}
\setbeamercolor{itemize item}{fg=CustomOrange}
\setbeamercolor{itemize subitem}{fg=CustomOrange}
\setbeamercolor{enumerate item}{fg=CustomOrange!90!black}
\setbeamercolor{enumerate subitem}{fg=CustomOrange!90!black}
\setbeamercolor{description item}{fg=CustomOrange}
\setbeamercolor{description subitem}{fg=CustomOrange}
\setbeamerfont{footline}{size=\tiny}

\setbeamersize{text margin left=0.4cm, text margin right=0.4cm}


% METADADOS ============================


\subtitle{Minicurso de Arduino}
\author{}
\institute[Facom/UFMS]{\vspace{-\bigskipamount}
	\begin{columns}[c, onlytextwidth]
		\column{0.6\linewidth}
			Faculdade de Computação -- Facom\bigskip\\
			Universidade Federal de Mato Grosso do Sul -- UFMS
		\column{0.22\linewidth}
			\includegraphics[width=\linewidth]{facomGraph.png}
	\end{columns}
}
\date{\the\year}


% PACOTES ==============================


\usepackage[brazilian]{babel}
\usepackage[type={CC},modifier={zero},version={1.0},lang=brazilian]{doclicense}
\usepackage[cache=false, section]{minted}
\usepackage{graphicx}
\usepackage{multirow}


% MINTED ===============================


\usemintedstyle{perldoc}
\usemintedstyle[arduino]{arduino}

\setminted{
%  beameroverlays=true,
	bgcolor=white,
	highlightcolor=CustomYellow!20!white,
	autogobble=true,
	tabsize=2,
	fontsize=\footnotesize,
	baselinestretch=1,
	breaklines=true,
	breakbytoken=true,
	escapeinside=@@,
	mathescape=true
}


% COMPORTAMENTO ========================


\mode<handout>{\AtBeginSubsection{}}
\AtBeginSection{}
\AtEndDocument{\FinalFrame}

\graphicspath{{"../imagens/"}{imagens/}} % Diretórios de imagens geral e local, respectivamente.

\renewcommand{\arraystretch}{1.1} % Aumenta o espaçamento entre linhas de tabelas em 10%.
